\documentclass[10pt]{article}
\usepackage{amsmath}
\usepackage{setspace}
\usepackage{hyperref}
\usepackage{booktabs}


\setlength{\textheight}{9in} \setlength{\topmargin}{-.5in}
\setlength{\textwidth}{6.5in} \setlength{\oddsidemargin}{0in}
\setlength{\evensidemargin}{0in}

\title{Syllabus \\ COMP 210 \\ Object-Oriented Programming}
\author{ }
\date{Spring 2017}

\begin{document}
\maketitle

\section{Logistics}
\begin{itemize}
\item \textbf{Where: }
\begin{itemize}
\item Class: Center for Science and Business, Room 303
\item Lab: Center for Science and Business, Room 309
\end{itemize}
\item \textbf{When: }
\begin{itemize}
\item Class: MWF 1--1:50pm
\item Lab: Th 2--4pm
\end{itemize}
\item \textbf{Instructor: } James \textit{Logan} Mayfield
\begin{itemize}
\item \textit{Office: } Center for Science and Business, Room 344
\item \textit{Phone: } 309-457-2200 % chktex 8
\item \textit{Email: } lmayfield \textit{at} monmouthcollege \textit{dot} edu
\item \textit{Office Hours: } Monday and Tuesday 3--4pm. Thursday 9--10am. By Appointment.
\item \textbf{Website: } \url{http://jlmayfield.github.io/}
\end{itemize}
\item \textbf{Website: } \url{http://jlmayfield.github.io/teaching/COMP210/}
\item \textbf{Credits: } 1 course credit
\end{itemize}
\emph{Note: This Syllabus is subject to change based on specific class needs. Deviations from the syllabus will be discussed in class.}


\section{Texts}

This course will be run largely from course notes, which are available from the course website, and online references. The standard, Oracle Provided, Java references and tutorials will be utilized throughout the semester.
\vspace{.25in}

%insert text #1
Oracle. \textit{Java SE APIs and Documentation}. 2013.
\newline
\url{http://www.oracle.com/technetwork/java/javase/documentation/api-jsp-136079.html}
\vspace{.25in}

Oracle. \textit{Java 7 API}. 2013.
\newline
\url{http://docs.oracle.com/javase/7/docs/api/overview-summary.html}


\section{Programming Environment}

All programs written in this course are required to \emph{compile and run} on a Linux computer with the latest JAVA JDK/JRE and the Eclipse IDE\@.  Specifically, they should be able to run on Church, the departmental server.  All of the software needed is available on Church and can be accessed via a VNC desktop session. While development is not required on these environments, \emph{failure to properly port a program to the required environment could result in a program not compiling correctly when it is being graded.}  All software for this course is available free of charge from Oracle and the Eclipse Foundation.
\begin{itemize}
\item \emph{Oracle JAVA JDK/JRE:}  Get the Java SE JDK\@.
\newline \url{http://www.oracle.com/technetwork/java/javase/downloads/index.html}
\item \textit{Linux OpenJDK} This is an open-source version of the JDK\@.  If you're running Linux, this might be easier for your to install and shouldn't cause you any problems.\newline
\url{http://openjdk.java.net/install/index.html}
\item \emph{Eclipse IDE:} Get either the Java Developers or Classic version. \newline
\url{http://www.eclipse.org/downloads/}
\end{itemize}


\section{Description and Content}

In this course, students will extend the \textit{Design Recipe} ideas to explore the paradigm of Object-Oriented programming (OOP) using the Java Programming language.  As is true with all programming paradigms, OOP constitutes not just a set of tools for writing programs but a way of thinking and reasoning about the structure of programs and computations in general.   The Object-Oriented paradigm draws on all the computing and programming concepts students studied in the introductory sequence.

\subsection{Content}

In this course students will explore the Object-Oriented paradigm for programming. Just as the shift from Functional programming in Racket to Imperative programming in C++ required a change in perspective, so to will the shift to OOP in Java.  Students will explore the conceptual foundations of OOP as well as modern OOP in Java.

\begin{itemize}
\item Designing Object-Oriented Class Hierarchies
\begin{itemize}
\item Classes and Class Hierarchies
\item Abstract Classes
\item Generics
\end{itemize}
\item Basic UML-style Diagrams
\item Developing GUI-Based Programs using Model-View-Controller (MVC)
\item Basic OO Design Patterns
\end{itemize}


\section{Expectations and Policies}

Students are expected to carry themselves in a mature and professional manner in this course. Towards this end, there are a few classroom policies by which every student is expected to abide.
\begin{itemize}

\item \textit{Late Assignments: } In general, late assignments will \textit{not} be accepted.  Students who feel they have a justified reason for submitting an assignment late may set up an appointment to meet with the instructor and plead their case.  Students are more likely to get extensions on assignments when they are asked for in advance rather than the day the assignment is due.

\item \textit{Attendance: } \textbf{Repeated absences and late arrivals to class will quickly reduce a student's participation grade to zero.}  The occasional late arrival or missed class is one thing, but being habitually late and regularly missing classes is disruptive and not fair to your classmates.

\item \textit{Participation: }  Cellphone and computer usage in class for non-class related activities is strongly discouraged.  All devices should be set to silent when in class.  If a student's usage of technology becomes a distraction to their classmates or the instructor, then that student's participation grade will suffer.  If the instructor or a classmate has to inform a student that they're being a distraction, then their use of technology has already gone too far.  When in doubt, err on the side of caution.

\item \textit{Quality of Work:} There are several minimal requirements that course assignments must meet.
\begin{itemize}
\item \textit{Electronic Submissions:}  Most work will be handed in electronically.  It is the student's responsibility to know and understand the system for doing so and to be sure that all their work has been properly submitted. Not following the instructions for assignment submission can mean an assignment does not get submitted and will be considered late.

\item \textit{Staples:} Printed assignments that take up more than one page must be stapled.  Multi-page  assignments lacking staples will either be returned to the student to be stabled ASAP or points will be deducted.

\item \textit{Neatness:}  Students should make every attempt to make their work neat and orderly. When doing hand written problems or exams, label problems, avoid excessive scratching out of mistakes (use a pencil if corrections are to be expected). All computer code should adhere to a clean and consistent style such that it's structure is easy to read with the human eye. Use indentation and spacing when expected and align all parenthesis and braces accordingly. Break up lines of code and comments that are longer than 70--80 characters to avoid wrapping when printing. Finally, do not be afraid to use comments to explain or label parts of the code.

\item \textit{Show Work:} Rarely are answers alone sufficient for full credit.  Show your work whenever prudent.  If you're unsure if work is needed, \textit{ask!}
\end{itemize}

\end{itemize}


\subsection{Collaboration}

In general, students are encouraged to make use of the resources available to them.  This means it is OK to seek help from a friend, the tutor, the instructor, the internet, etc.  However, \textit{copying of answers and any act worthy of the label of ``cheating'' or ``plagiarism'' is never permissible!. Students should always be able to reproduce an answer on their own, and if they cannot then they likely \textbf{do not really known the material.}} All of the Monmouth College rules on academic dishonesty apply.  A student found in violation of the rules should be prepared to face the consequences of their actions. If a student needs help understanding the rules, then please seek out the instructor before doing something that might violate academic honesty policies.

\section{Grades}

This courses uses a standard grading scale.  Assignments and final grades will not be curved except in rare cases when its deemed necessary by the instructor.  Percentage grades translate to letter grades as follows:

\begin{center}
\begin{small}
\begin{tabular}{lcl}
Score & & Grade \\ \toprule
94--100 & & A \\
90--93 & & A- \\
88--89 & & B+ \\
82--87 & & B \\
80--81 & & B- \\
78--79 & & C+ \\
72--77 & & C \\
70--71 & & C- \\
68--69 & & D+ \\
62--67 & & D \\
60--61 & & D- \\
0--59 & & F
\end{tabular}
\end{small}
\end{center}


You are always welcome to challenge a grade that you feel is unfair or calculated incorrectly.  Mistakes made in your favor will never be corrected to lower your grade.  Mistakes made not in your favor will be corrected.  \textit{Basically, after the initial grading your score can only go up as the result of a challenge.}

\subsection{Workload}
% number of/details on midterms, finals, project, homeworks, quizes, etc

The course workload is as follows:
\begin{center}
  \begin{tabular}{ll}
    Category & Number of Assignments \\ \toprule
    Labs & 10 \\
    Homework & 8--10 \\
    Projects & 2 \\
    Exams & 7
  \end{tabular}
\end{center}

Homework assignments will always either precede or follow a lab to prepare for a complete a lab respectively. Students are also encouraged to look at the text's review questions as the solutions are available online.  The instructor will also identify key problems from each chapter that make for good practice/study problems. There will be no dedicated midterm or final exam. There are just exams.  All exams but the very last will focus on material covered since the previous exam. The final exam will include a small number of cummalative questions but still focus on the most recently covered course material.

\subsubsection{Lab and Homework Grades}

Lab and homework assignments are graded on a simple 3 point scale. Grades are marked with, in decreasing order, a check-plus, check, or check-minus. Your final grade for these two assignment categories is then based off the respective averages and determined by the following chart.  Notice this chart lists the minimum average needed to achieve a particular letter grade.

\begin{center}
\begin{small}
\begin{tabular}{ll}
Assignment Avg. (Min) & Letter Grade \\ \toprule
2.8   & A  \\
2.75    & A- \\
2.5 & B+ \\
2.25    & B  \\
2   & B- \\
1.75    & C+ \\
1.5 & C  \\
1   & C- \\
0.75    & D  \\
0.5  & F
\end{tabular}
\end{small}
\end{center}

\subsection{Grade Weights}

Your final grade is based on a weighted average of particular assignment categories.  You should be able to estimate your current grade based on your scores and these weights.  You may always visit the instructor \textit{outside of class time} to discuss your current standing.

\begin{center}
  \begin{tabular}{ll}
  Category & Weight \\ \toprule
    Exams & 45\% \\ %~6.43 each
    Projects & 25\% \\ %12.5 each
    Homework & 12.5\% \\ %1.5 -- 1.25 each
    Labs & 12.5\% \\ %1.25 each
    Participation & 5\%
  \end{tabular}
\end{center}


\subsection{Course Engagement Expectations}

The weekly workload for this course will vary by student but on average should be about 13 hours per week.  The follow tables provides a rough estimate of the distribution of this time over different course components for a 16 week semester.
\begin{center}
\begin{tabular}{lll}
Assignment Type & Total Time & Time/week \\ \toprule
Lectures+Labs &      & 4 hours/week \\
Homework & 45 hours        & 3 hours/week \\
Exam Study Time & 16 hours  & 1 hours/week \\
Projects & 45 hours        & 3 hours/week \\
Reading+Unstructured Study & & 2 hours/week \\
\bottomrule
& & 13 hours/week
\end{tabular}
\end{center}

\subsection{Calendar}

The following calendar should give you a feel for how work is distributed throughout the semester.  \textit{This calendar is subject to change based on the circumstances of the course.}

\begin{center}
\begin{tabular}{lll}
Week & Dates & Assignments \\ \toprule
1 & 1/16--1/20 &  Lab 1.  \\
2 & 1/23--1/27 & Lab 2.   \\
3 & 1/30--2/3 & Lab 3. Exam 1. \\
4 & 2/6--2/10 & Lab 4. Exam 2.\\
5 & 2/13--2/17 & Project 1 Hwk. \\
6 & 2/20--2/24 & Project 1. \\
7 & 2/27--3/2 & Exam 3.
\textit{No Class Friday.} \\
  & 3/6--3/10 & \textit{Spring Break} \\
8 & 3/13--3/17 & Lab 5. \\
9 & 3/20--3/24 & Lab 6. Exam 4. \\
10 & 3/27--3/31 & Lab 7.  \\
11 & 4/3--4/7 & Lab 8. Exam 5.  \\
12 & 4/10--4/13 & Lab 9. Exam 6. \textit{No Class Friday.}   \\
13 & 4/18--4/21 & Project 2 Homework. \textit{No Class Monday.}    \\
14 & 4/24--4/28 &  Project 2. \\
15 & 5/1--5/3 & Lab 10.  \\ \midrule
  & 5/9 & Exam 7 \textit{(11:30--2:30pm)}  \\
\end{tabular}
\end{center}

\end{document}
