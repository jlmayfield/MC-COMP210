\documentclass[]{tufte-handout}
\usepackage{amsmath, amssymb, amsthm}
\usepackage{hyperref}

\title{COMP210 --- Lab 3}
\author{}
\date{ Spring 2017 }

\begin{document}
\maketitle

\begin{abstract}
In the third lab and homework you'll be \textit{refactoring}\sidenote{\url{https://en.wikipedia.org/wiki/Code_refactoring}} your design from homework 2 to include the use of abstract classes and class extension. You can use your solution to homework 2 or the one given by the instructor\sidenote{see /home/comp210/sp17/} to complete this assignment.
\end{abstract}

\section*{Lab3}

Your goal for the lab is to redraw the UML diagram\sidenote{the pdf is also on the server} for a new class hierarchy design that lifts all of the shared media file implementation into an abstract class called \textit{AbstractMediaFile}. For the sake of clarity, restate all Media interface methods in the class in which it can/will be implemented. Anything implemented in \textit{AbstractMediaFile} doesn't need to be restated in the concrete Media types. Put another way, you should propagate interface methods down the hierarchy until it's implemented.

If by the end of lab you haven't shown the instructor your updated diagram, then do so to get lab credit\sidenote{You should probably be checking your progress as you go though\ldots}.

\section*{Hwk3}

\begin{center}
\textbf{Submit as \textit{hwk3} via handin no later than 1pm on 2/8}
\end{center}

Now refactor the the code to meet the new design. Do not modify your lab 2 code directly, but instead work off a copy of the project. To copy an existing project in Eclipse you can highlight the project, hit Ctrl-C to copy, then hit Ctrl-V to ``paste'' the project. You'll be given the option to rename the project at this point.\sidenote{You might still want to rename packages}. In theory, you should not need to modify any of the concrete Media type tests as the change that we're making is to the implementation and not the overall interface.\sidenote{we're changing \textit{How} not \textit{What}}. If and when tests need to be rewritten, then you should think about whether or not you're re-factoring isn't also redesigning the overall interface. The example code from lecture notes 5 should act as a good guide for this exercise.


\end{document}
