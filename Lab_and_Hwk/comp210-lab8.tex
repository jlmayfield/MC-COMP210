\documentclass[]{tufte-handout}
\usepackage{amsmath,amssymb,amsthm,color}
\usepackage{multirow}
\usepackage{booktabs}
\usepackage{framed}
\usepackage[pdftex]{graphicx}

  
\title{COMP 210 - Lab 8 \& Homework 7}
\date{Spring 2016}

\begin{document} 
\maketitle

\begin{abstract}
In this lab you'll covert our unit converter GUI into a Model-View-Controller design with the Observer-Observable pattern as set out in the tutorial by Joesph Mack that we looked at in class\sidenote{\url{http://www.austintek.com/mvc/}}. 
\end{abstract}

\section{Lab 8}

For lab you should get started on the redesign and be certain you get the nuts and bolts of the original converter GUI program as well as the MVC example. There are three good options for starting this redesign:
\begin{enumerate}
\item Start from the Converter GUI program and rework that code into MVC
\item Start from the MVC example and rework that into the Converter GUI program
\item Start from a blank page and work off the other two as a reference
\end{enumerate}

The first two options are appealing in you get to work starting from something that does some of what you need but not all of it. You \textit{might} get done faster this way. The third option forces you to repeat what you've read and while it might take longer, the repetition is likely to lead to a deeper understanding of the principles at play.  I will not force one way of working on you, but I do recommend option three for the sake of long term learning and understanding. At some point you'll start one of these programs from a blank page. It might as well be now.   


\section{Homework 7}

\begin{center}
\textbf{Due by class time on Monday}
\end{center}

Finish the program and submit the completed source code as assignment \textit{hwk7}.


\end{document}