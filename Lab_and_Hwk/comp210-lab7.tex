\documentclass[]{tufte-handout}
\usepackage{amsmath, amssymb, amsthm}
\usepackage{hyperref}
\usepackage{framed}
\usepackage{listings}
\usepackage{color}

\definecolor{dkgreen}{rgb}{0,0.6,0}
\definecolor{gray}{rgb}{0.5,0.5,0.5}
\definecolor{mauve}{rgb}{0.58,0,0.82}

\lstset{
  frame=tb,
  language=Java,
  aboveskip=3mm,
  belowskip=3mm,
  showstringspaces=false,
  columns=flexible,
  basicstyle={\small\ttfamily},
  numbers=left,
  stepnumber=2,    
  firstnumber=1,
  numbersep=5pt,
  numberfirstline=true,
  numberstyle=\tiny\color{black},
  keywordstyle=\color{blue},
  commentstyle=\color{dkgreen},
  stringstyle=\color{mauve},
  breaklines=true,
  breakatwhitespace=true,
  tabsize=3
}


\title{COMP210 - Lab 7 \& Homework 6}
\author{}
\date{Spring 2016}



\begin{document}
\maketitle

\begin{abstract}
For this lab you'll start working through the basics of GUI implementation in Java.  Your task is the close, critical examination of basic examples in order to learn the high-level nuts and bolts of the Java Swing library.
\end{abstract}

\section{The Swing Tutorial}

Oracle\sidenote{by way of the acquisition of Sun} provides a wealth of tutorials on Java programming.  Their ``Swing Tutorial''\sidenote{\url{http://docs.oracle.com/javase/tutorial/uiswing/}} introduces the basics of GUI programming in Java.  Your task for the day is simple: \textit{follow through and do the tutorial.} Be sure to read the section below about using Eclipse for the tutorial. Simply reading the tutorial is insufficient.  You should download, run, and then modify the ``Hello World'' program and the temperature converter program. Modifications you should explore at this point are:
\begin{itemize}
\item Changing the text in the hello world program
\item Changing the conversion (maybe Bytes to bits rather than temperatures?) on the temperature converter. This includes relabeling everything.
\item Add new or different components to the converter (maybe allow for many different conversions via radio button or drop down selection?)
\item Play with different layouts
\item Change the initial size and placement of the GUI window. 
\end{itemize}
At the end of lab, use handin to submit your modified temperature converter program as assignment \textit{lab7}.

\subsection*{Eclipse and the Tutorial}

The tutorial doesn't use Eclipse. You're directed to either build and run Java at the CLI or use the Netbeans IDE. You should use Eclipse for all of the examples.  The tutorial provides all the java source files, you'll need to create one or more Eclipse projects and import the files appropriately. Be sure to pay attention to packages\sidenote{packages correspond to folders}. The temperature converter program is an example of code generated by a program. In this case a WYSIWYG\sidenote{what you see is what you get} program in Netbeans was used to visually layout the GUI and what you see is the code written by the visual layout program. For homework you'll be rewriting this into something more human readable. 

\section{Homework 6}

\begin{center}
\textbf{Due by class on Wednesday April 6th}
\end{center}

Your homework is to modify the temperature converter\sidenote{or your revision of that code} into a more human readable style. This boils down to three things:
\begin{enumerate}
\item Add explicit \textit{this} in place of implicit \textit{this}.
\item Add imports and remove explicit package+class names from the code
\item Use variables and sequential statements instead of chained method calls.\sidenote{foo.A().B().C().D()}
\end{enumerate}
Fixing these things forces you to explore the documentation for called methods to determine their return types so that you can create variables to store that returned value or to decide that the method is a mutatator that returns a reference to the mutated object. In the later case no intermediate variable is needed, we can simply call the mutator for effect as a single statement.  In doing this, you'll gain a more complete picture of the hierarchy of classes and interfaces that underlies Java's GUI framework. Submit the rewritten code as assignment \textit{hwk 6}. An optional exercise you might undertake is to draw out a UML hierarchy diagram for the program and include the parts of the library used in the program. 


\subsection*{Example Rewrite}

In CelciusConverter.java you see the code shown in figure \ref{orig}\sidenote{This code doesn't appear in this exact order}.
\begin{figure}[ht]
\begin{lstlisting}
fahrenheitLabel = new javax.swing.JLabel();
convertButton = new javax.swing.JButton();
javax.swing.GroupLayout layout = new javax.swing.GroupLayout(getContentPane());

layout.createSequentialGroup()
   .addComponent(convertButton)
   .addPreferredGap(javax.swing.LayoutStyle.ComponentPlacement.RELATED)
   .addComponent(fahrenheitLabel)
\end{lstlisting}
\label{orig}
\caption{Netbeans Generated Code}
\end{figure}

This code can be re-written as seen in figure \ref{revise} by add \textit{this} where is was implied and using variables to store intermediate values of chained methods. For this assignment we'll excuse terrible variable names like \textit{sgroup}.
\begin{figure}[ht]
\begin{lstlisting}
this.fahrenheitLabel = new JLabel();
this.convertButton = new JButton();
GroupLayout layout = new javax.swing.GroupLayout(this.getContentPane());

GroupLayout.SequentialGroup sgroup = layout.createSequentialGroup();
sgroup.addComponent(convertButton);
sgroup.addPreferredGap(javax.swing.LayoutStyle.ComponentPlacement.RELATED);
sgroup.addComponent(fahrenheitLabel);
\end{lstlisting}
\label{revise}
\caption{Revised Code}
\end{figure}


Finally, for the code in figure \ref{revise} to work we need to add the imports shown in figure \ref{ports} to the top of CelciusConverter.java. 
\begin{figure}[ht]
\begin{lstlisting}
import javax.swing.GroupLayout;
import java.awt.Container;
import javax.swing.JLabel;
import javax.swing.JButton;
\end{lstlisting}
\label{ports}
\caption{Imports for Revised Code}
\end{figure}




\end{document}

