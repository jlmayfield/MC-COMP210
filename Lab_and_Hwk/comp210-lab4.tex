\documentclass[nobib]{tufte-handout}
\usepackage{amsmath,amssymb,amsthm,color}
\usepackage{multirow}
\usepackage{booktabs}
\usepackage[pdftex]{graphicx}


\title{COMP 210 \\ Lab 4}
\date{Spring 2017}

\begin{document}
\maketitle

\begin{abstract}
In this lab we'll continue work on our functional binary tree design from class. To let us focus on trees we'll design a Binary tree of String objects.
\end{abstract}

Begin by stubbing out the hierarchy containing a BinTree interface, AbstractNode abstract class, and MTTree, Leaf, and TreeNode concrete classes. For each class include:
\begin{itemize}
  \item \textit{private} default constructors and by-field constructors. Abstract classes should use protected constructors.
  \item public static factory methods for object construction\sidenote{further discussion in lab}
  \item Accessors for each field. No mutators.
  \item equals and hashCode methods
  \item toString
\end{itemize}

The interface should initially include the following methods\sidenote{see lecture notes 8 if you need a refresher on these }:
\begin{itemize}
  \item A method for computing tree size
  \item A method for computing tree height
  \item A method which fills an array\sidenote{\url{https://docs.oracle.com/javase/tutorial/java/nutsandbolts/arrays.html}} of the Strings with the Strings contained in the tree in the order they'd be visited by an inorder traversal. Your design for this method should allow users to use a pre-allocated array and not require any extra arrays to be allocated in memory.
\end{itemize}

Lab time should be used for documenting, declaring, stubbing, and developing tests for the above listed methods. When that is complete or at the end of lab, use handin to submit your work as \textit{lab4}. Implementation should then be completed for assignment \textit{hwk4} which is due by \textbf{Monday 2/13}.

\end{document}
