\documentclass[10pt]{article}
\usepackage{amsmath}
\usepackage{setspace}
\usepackage{hyperref}


\setlength{\textheight}{9in} \setlength{\topmargin}{-.5in}
\setlength{\textwidth}{6.5in} \setlength{\oddsidemargin}{0in}
\setlength{\evensidemargin}{0in}

\title{Syllabus - COMP 210 - Object-Oriented Programming}
\author{ James \textit{Logan} Mayfield }
\date{Spring 2016}

\begin{document}
\maketitle

\section{Logistics}
\begin{itemize}
\item \textbf{Where: } 
\begin{itemize}
\item Class: Center for Science and Business, Room 303
\item Lab: Center for Science and Business, Room 309
\end{itemize}
\item \textbf{When: } 
\begin{itemize}
\item Class: MWF 1-1:50pm
\item Lab: Th 2-4pm
\end{itemize}
\item \textbf{Instructor :} James \textit{Logan} Mayfield
\begin{itemize}
\item \textit{Office: } Center for Science and Business, Room 344
\item \textit{Phone: } 309-457-2200
\item \textit{Email: } lmayfield \textit{at} monmouthcollege \textit{dot} edu
\item \textit{Office Hours: } By Appointment.
\end{itemize}
\item \textbf{Credits: } 1 course credit
\end{itemize}
\emph{Note: This Syllabus is subject to change based on specific class needs. Deviations from the syllabus will be discussed in class.}


\section{Text}

This course will be run largely from course notes and online references.  The text \textit{How to Design Classes} is the basis for much of the early material and assignments.  The standard, Oracle Provided, Java references and tutorials will also be utilized throughout the semester.
\vspace{.25in}

%insert text #1
Felleisen, M., et. al., \textit{How To Design Classes}. MIT Press. DRAFT. 2008
\newline
\url{http://www.ccs.neu.edu/home/matthias/htdc.html}
\vspace{.25in}

Oracle. \textit{Java SE APIs and Documentation}. 2013. 
\newline
\url{http://www.oracle.com/technetwork/java/javase/documentation/api-jsp-136079.html}
\vspace{.25in}

Oracle. \textit{Java 7 API}. 2013.
\newline
\url{http://docs.oracle.com/javase/7/docs/api/overview-summary.html}


\section{Programming Environment}
All programs written in this course are required to \emph{compile and run} on a Linux computer with the latest JAVA JDK/JRE and the Eclipse IDE.  Specifically, they should be able to run on Church, the departmental server.  While development is not required on these environments, \emph{failure to properly port a program to the required environment could result in your program not compiling correctly when it is being graded.}  All software for this course is available free of charge from Oracle and the Eclipse Foundation.
\begin{itemize}
\item \emph{Oracle JAVA JDK/JRE:}  Get the Java SE JDK. 
\newline \url{http://www.oracle.com/technetwork/java/javase/downloads/index.html}
\item \textit{Linux OpenJDK} This is an open-source version of the JDK.  If you're running Linux, this might be easier for your to install and shouldn't cause you any problems.\newline
\url{http://openjdk.java.net/install/index.html} 
\item \emph{Eclipse IDE:} Get either the Java Developers or Classic versions. \newline
\url{http://www.eclipse.org/downloads/} 
\end{itemize} 


\section{Description and Content}

In this course, students will extend the \textit{Design Recipe} ideas to explore the paradigm of Object-Oriented programming (OOP) using the Java Programming language.  As is true with all programming paradigms, OOP constitutes not just a set of tools for writing programs but a way of thinking and reasoning about the structure of programs and computations in general.   The Object-Oriented paradigm draws on all the computing and programming concepts students studied in the introductory sequence.

\subsection{Content}

In this course students will explore the Object-Oriented paradigm for programming. Just as the shift from Functional programming in Racket to Imperative programming in C++ required a change in perspective, so to will the shift to OOP in Java.  Students will explore the foundations of OOP as well as modern OOP in Java. 

\begin{itemize}
\item Designing Object-Oriented Class Hierarchies
\begin{itemize}
\item Classes and Class Hierarchies [HtDC1-16]
\item Abstract Classes [HtDC18-19]
\item State Encapsulation [HtDC20]
\item Generics and Sub-typing [HtDC30-32]
\end{itemize}
\item Basic UML-style Diagrams
\item Developing GUI-Based Programs using Model-View-Controller (MVC)
\item Basic OO Design Patterns
\end{itemize}


\section{Expectations and Policies}
The expectations for students in this course are not at all unreasonable.  To avoid any confusion, they are listed here.  These aren't necessarily rules but rather guidelines for how you should conduct yourself in this class.  Strict rules will result from these expectations and will be covered later.
\begin{itemize}
\item Be respectful of others.  Don't create unnecessary distractions.  Turn cell phones off, on silent or leave them in the dorm.  Class time is not the time for checking email, surfing the web and IMing.  \textit{Come to class ready and interested in learning and if you're not, don't behave in such a way that prevents others from doing so.}
\item You're in college.  College is meant to provide an education.  Therefore, you are, for all intents and purposes, a \textit{professional student}.  Your work should reflect a solid level of professionalism and be neat and orderly.  Take the extra time to make it presentable.  Crumpled papers with various liquid stains on them are not presentable.  Think of the instructor as your boss and that the quality of your paycheck depends on the quality of the work.  \textit{You don't have to always love the work you do, but you should always do it to the best of your capabilities.}
\item Attending class is not by itself sufficient for learning the material.  You're expected to read the sections of the text as they are covered in class.  You are encouraged to go beyond the material.  Make use of available resources such as tutors and the high availability of your instructor.  \textit{Don't expect to get an A just by showing up and doing the least amount of work that you can.}
\end{itemize}

There are several strict policies that result from these expectations.  In the case of these items, they are rules and you are responsible for understanding and abiding by them.
\begin{itemize}
\item \textit{Late Assignments: } In general, \textbf{late assignments will \textit{not} be accepted}.  If you feel you have a justified reason for the assignment being late you may set up an appointment to meet with the instructor and plead your case.  Situations beyond your control are understandable and exceptions can and will be made.
\item \textit{Attendance: } \textbf{Repeated absences and late arrivals to class will quickly reduce your participation grade to zero.}  The occasional late arrival or missed class is one thing, but being habitually late and regularly missing classes is disruptive and not fair to your classmates.  
\item \textit{Participation: }  Cellphone usage in class is not allowed, this includes text messages.  Turn off the ringers or leave them at home.  Computer usage is limited to activities in support of the course.  This does not include IMs, Facebook, checking email, general web surfing, poker, fantasy sports leagues, forum trolling, mine sweeper, etc.  This behavior is rude and can be a real distraction to others.  Repeated failure to abide by this policy will have a negative effect on your grade.  
\item \textit{Quality of Work:} There are several minimal requirements that your assignments must meet.
\begin{itemize}
\item \textit{Staples - } Assignments that take up more than one page must be stapled.  Unstapled assignments will either be returned to you to be stabled ASAP or points will be deducted.  
\item \textit{Neatness - }  Make every attempt to make your work neat and orderly:  label problems, avoid excessive scratching out of mistakes (use pencil if you are prone to errors) and if you use spiral bound paper tear off the edges. 
\item \textit{Show Work - } Rarely are answers alone sufficient for full credit.  Show your work whenever prudent.  If you're unsure if work is needed, \textit{ask!}
\end{itemize}
\end{itemize}

\subsection{Collaboration}

In general, you are encouraged to make use of the resources available to you.  This means it is OK to seek help from a friend, tutor, instructor, internet, etc.  However, \textit{copying of answers and any act worthy of the label of ``cheating'' is never permissible!}  It is understandable that when you work with a partner or a group that the resultant product is often extremely similar.  This is acceptable but be prepared to be asked to defend your collaborations to the instructor.  \textit{You should always be able to reproduce an answer on your own, and if you cannot you likely \textbf{do not really known the material.}} 
\begin{itemize}
\item When assignments are meant to be done in groups, you will be directed to turn in one set of solutions per group.
\item All other assignments should represent your own work and effort.
\end{itemize}
All of the Monmouth College rules on academic dishonesty apply.  If you violate the rules be prepared to face the consequences of your actions.  

\section{Grades}

This courses uses a standard grading scale.  Assignments and final grades will not be curved except in rare cases when its deemed necessary by the instruction.  Percentage grades translate to letter grades as follows:

\begin{center}
\begin{small}
\begin{tabular}{lcl}
Score & & Grade \\ \hline
94-100 & & A \\
90-93 & & A- \\
88-89 & & B+ \\
82-87 & & B \\
80-81 & & B- \\
78-79 & & C+ \\
72-77 & & C \\
70-71 & & C- \\
68-69 & & D+ \\
62-67 & & D \\
60-61 & & D- \\
0-59 & & F 
\end{tabular}
\end{small}
\end{center}

You are always welcome to challenge a grade that you feel is unfair or calculated incorrectly.  Mistakes made in your favor will never be corrected to lower your grade.  Mistakes made not in your favor will be corrected.  \textit{Basically, after the initial grading your score can only go up as the result of a challenge.}

\subsection{Lab and Homework Grades}

Lab and homework assignments are graded on a simple 3 point scale.  Your final grade for these two assignment categories is then based off the respective averages and determined by the following chart.  Notice this chart lists the minimum average needed to achieve a particular letter grade.  

\begin{center}
\begin{small}
\begin{tabular}{ll}
Assignment Avg. (Min) & Letter Grade \\ \hline
2.8   & A  \\
2.75    & A- \\
2.5 & B+ \\
2.25    & B  \\ 
2   & B- \\
1.75    & C+ \\
1.5 & C  \\
1   & C- \\
0.75    & D  \\
0.5  & F 
\end{tabular}
\end{small}
\end{center}



\subsection{Workload}
% number of/details on midterms, finals, project, homeworks, quizes, etc
The course workload is as follows:
\begin{itemize}
\item 10 Labs
\item 7 Homework Assignments
\item 5 Quizzes
\item 2 Projects
\item 1 Final Exam
\item 1 Midterm Exam
\end{itemize}

Homework assignments will generally be attached to labs as either a pre-lab or post-lab assignment. You can, therefore, expect them to be assigned in conjunction with the majority of the labs.


\subsection{Grade Weights}
Your final grade is based on a weighted average of particular assignment categories.  You should be able to estimate your current grade based on your scores and these weights.  You may always visit the instructor \textit{outside of class time} to discuss your current standing.  
\begin{itemize}
\item Quizzes - 25\%
\item Projects - 25\%
\item Final - 12.5\%
\item Midterm - 12.5\%
\item Homework - 10\%
\item Labs - 10\%
\item Participation \& Attendance - 5\%
% add more if needed
\end{itemize} 


\subsection{Course Engagement Expectations}

The weekly workload for this course will vary by student but on average should be about 12.5 hours per week.  The follow tables provides a rough estimate of the distribution of this time over different course components for a 15 week semester. 
\begin{center}
\begin{tabular}{|l|l|l|}
\hline
Lectures+Labs+Final &      & 4.2 hours/week \\ 
Homework & 30 hours        & 2 hours/week \\
Exam Study Time & 8 hours  & 0.5 hours/week \\ 
Quiz Study Time & 10 hours & 0.6 hours/week \\
Projects & 48 hours        & 3.2 hours/week \\
Reading+Unstructured Study & & 2 hours/week \\
\hline 
& & 12.5 hours/week \\ 
\hline
\end{tabular}
\end{center}


\subsection{Calendar}

The following calendar should give you a feel for how work is distributed throughout the semester.  \textit{This calendar is subject to change based on the circumstances of the course.}

\begin{center}
\begin{tabular}{|l|l|l|}
\hline 
Week & Dates & Assignments \\
\hline
1 & 1/11 - 1/15 &  Lab 1.  \\
\hline
2 & 1/18 - 1/22 & Homework 1. Lab 2. Quiz 1. \\
\hline
3 & 1/25 - 1/29 & Homework 2. Lab 3.  \\
\hline
4 & 2/1 - 2/5 & Homework 3. Lab 4. Quiz 2.\\
\hline
5 & 2/8 - 2/12 & Homework 4. Lab 5. Quiz 3.  \\
\hline
6 & 2/15 - 2/19 & Lab 6. \\
\hline
7 & 2/22 - 2/26 & Homework 5. Lab 7.  Quiz 4  \\
\hline
8 & 2/29 - 3/4 & Lab 8. Midterm Exam. \\
\hline 
SPRING BREAK & 3/7 - 3/11 & \\
\hline
9 & 3/14 - 3/18 & Project 1 Homework.  \\
\hline
10 EASTER BREAK (Fr). & 3/21 - 3/24 & Project 1.  \\
\hline
11 EASTER BREAK (Mo). & 3/29 - 4/1 & Lab 9.  \\
\hline
12 & 4/4 - 4/8 & Homework 6. Lab 10.  \\
\hline
13 & 4/11 - 4/15 & Homework 7. Quiz 5.   \\
\hline
14 & 4/18 - 4/22 & Project 2 Homework.  \\
\hline
15 & 4/25 - 4/29 & Project 2.  \\ 
\hline
16 & 5/2 - 5/4 &   \\
\hline
Final's Week & 5/7 (11:30am-2:30pm) & Final Exam.   \\ 
\hline
\end{tabular}
\end{center}

\end{document}